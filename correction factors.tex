Sürecin örneklem ortalamasını tahmin etmek için $\bar{\bar{x}}$ kullanılabilir ve değeri $16,2680$ olacaktır.\\

Süreç standart sapmasını hesaplamak için hem örneklem standart sapmasını hem de range değerleri ile hesaplamalar yapıldığında,\\


${\hat{\sigma}}_x = \frac{\bar{s}}{c_4} = \frac{0,1757}{0,9400} = 0,1869$
ve
${\hat{\sigma}}_x = \frac{\bar{R}}{d_2} = \frac{0,475}{2,326} = 0,2042$\\

Birbirine yakın ama aynı olmayan iki farklı süreç standart sapması hesaplandı. Bunun sebebi örneklem boyutu, seçilen örneklem istatistiğin ve yanlılık düzeltme katsayıları arasındaki farktır.

\subsection{Tahmincilerin yansızlığı ve beklenen değer operatörü ile hesaplamalar}

Örneklem ortalaması ve varyansı olan $\bar{x}$ ve $s^{2}$ popülasyon ortalaması ve varyansı olan $\mu$ ve $\sigma^{2}$ için yansız bir tahmincidir.\\

İstatistiksel olarak $E(\bar{x}) = \mu$ ve $E(\bar{s^{2}}) = \sigma^{2}$ şeklinde gösterilir. Çünkü $T(X) = \hat{\theta}$ eğer ki $\theta$ için bir tahminci ise o zaman bu tahmincinin $\theta$ için sapması, farkı ya da yanlılığı $bias(\hat{\theta}) = E_{\theta}(\hat{\theta}) - \theta$ olacaktır. Yani, yanlılık için beklenen değer ile hesaplamalar yapmamız gerekmektedir.\\

Fakat, örneklem standart sapması olan $s$ popülasyon standart sapması olan $\sigma$ için yansız bir kesirci değildir. Bu sebeple $E(s) \neq \sigma$ olacaktır.

Örneklem standart sapmasının neden popülasyon standart sapmasına eşit olmadığını ispatlamak için Jensen’s İnequality ile işlemler yapalım.\\

Eğer $f : I \to \mathbb{R}$ ve tüm $x \in I$ için $f''(x) \le 0$ konkav bir fonksiyon ve $X$ integrallenebilir bir rasgele değişken olmak üzere, 
\begin{itemize}[label*={}]
	\item 
	Jensen's Inequality $E(f(X)) \le f(E(X)) $
	\item 
	$f(s) = \sqrt{s^{2}}$
	\item
	$E(\sqrt{s^{2}}) \le \sqrt{E(s^{2})}$ ve \text{$E(s^{2}) = \sigma^{2}$, $\sqrt{\sigma^{2}} = \sigma$ olduğu için}
	\item 
	$E(s) \le \sigma$
\end{itemize}

Bu ispattan da anlaşılabileceği gibi, örneklem varyansının karekökü popülasyon varyansına her zaman eşit değildir. Sadece parametrenin gerçek değerine eşit olursa birbirlerine eşit olabilirler. Bu sebeplerden ötürü $E(s) = \sigma$ doğru bir ifade değildir. Bu yüzden örneklem boyutu $2\le n \le25$ olan süreçlerde düzeltme faktörü olarak kullanılan $c_4$, popülasyon parametresini için daha doğru sonuçlar verir. $E(s)=\left(\frac{2}{n-1}\right)^{\frac{1}{2}}\cdot \frac{\Gamma (n/2)}{\Gamma(\frac{n-1}{2})}\cdot \sigma =c_4\cdot \sigma$ ve populasyon tahmincisi olarak $\hat{\sigma }=\frac{\bar{s}}{c_4}$ şeklinde hesaplanır\\