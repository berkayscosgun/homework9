Soruya ilk yaklaşım olarak, bize verilen 5 elemanlı ve 20 deneyden oluşan Table 6E.6 verilerini kullanarak, ilerleyen hesaplamalarımızda gerekli olan istatistikleri önceden elde edebilmek amacıyla 'Table 6E.6.1' adlı yeni bir tablo oluşturacağız ve soru çözümlerinde ona başvuracağız. 

\renewcommand{\arraystretch}{1.2}

\begin{table}[h!]
	\centering
	\caption*{\bfseries Table 6E.6.1}
	\begin{tabular}{|c|c|c|c|}
		\hline
		\textbf{Sample} & \textbf{$\bar{x_i}$} & \textbf{$s_i$} & \textbf{$R_i$} \\
		\hline
		1 & 16,2000 & 0,2608 & 0,8000 \\
		2 & 16,1400 & 0,2059 & 0,5000 \\
		3 & 16,3000 & 0,1414 & 0,4000 \\
		4 & 16,2000 & 0,1673 & 0,5000 \\
		5 & 16,2200 & 0,1939 & 0,5000 \\
		6 & 16,3200 & 0,3311 & 0,9000 \\
		7 & 16,3000 & 0,1789 & 0,4000 \\
		8 & 16,1800 & 0,0748 & 0,2000 \\
		9 & 16,3400 & 0,1020 & 0,3000 \\
		10 & 16,3800 & 0,1720 & 0,5000 \\
		11 & 16,2400 & 0,1855 & 0,5000 \\
		12 & 16,3800 & 0,2926 & 0,8000 \\
		13 & 16,3200 & 0,1939 & 0,5000 \\
		14 & 16,3400 & 0,1020 & 0,3000 \\
		15 & 16,2400 & 0,1020 & 0,3000 \\
		16 & 16,2000 & 0,1095 & 0,3000 \\
		17 & 16,3000 & 0,0894 & 0,2000 \\
		18 & 16,2400 & 0,1855 & 0,5000 \\
		19 & 16,3000 & 0,1549 & 0,4000 \\
		20 & 16,2200 & 0,2713 & 0,7000 \\
		\cline{2-4}
	&$\bar{\bar{x}}=16,2680$&$\bar{s}=0,175$7&$\bar{R}=0,475$\\
		\hline
	\end{tabular}
	\caption*{}
\end{table}

Her bir deneydeki sabit gözlem sayısı olan n=5 bilgisi ile ihtiyacımız olacak tüm katsayıların sayısal değerlerini biliyoruz.  

Tablodaki değerlerin hesaplaması için: 

\begin{enumerate}[label=\Roman{enumi}.]
	

\item
$\bar{x} = \frac{1}{20}\sum_{i=1}^{20}\bar{x_i} = \frac{1}{20}(325,3600) = 16,268$
\item
$\bar{s}=\frac{1}{20}\sum _{i=1}^{20}\bar{s_i}=\frac{1}{20}\left(3,5147\right)=0,1757$
\item
$\bar{R}=\frac{1}{m}\sum _{i=1}^{20}R_i=\frac{1}{20}\left(9,500\right)=0,475$

\end{enumerate}

\cleardoublepage

Bu bilgilerin bize sundukları: 

\begin{enumerate}[label=\Roman{enumi}.]
	\item
	Her bir deney için farklı olan ortalamanın, toplam deney sayısına göre ortalaması 
	\item
	Her bir deney için farklı olan standart sapmalarının, toplam deney sayısına göre ortalaması 
	\item
	Her bir deneydeki en uç değerlerin farkının, toplam deney sayısına göre ortalaması 
\end{enumerate}

Öncelikle, $\bar{X}$ kontrol kartının grafiğini çizmeye çalışacağız. 

\begin{itemize}
	\item
	\centering
	$UCL=\bar{\bar{x}}+A_3\cdot \ \bar{s}=16,2680+\left(0,680\ast 0,1757\right)=16,5188$
	\item
	$CL\ =\ \bar{s}\ =\ 16,2680$
	\item
	$LCL=\bar{\bar{x}}-A_3\cdot \bar{s}=16,2680-\left(0,680\ast 0,1757\right)=16,0172$
\end{itemize}


